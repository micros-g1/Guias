\documentclass[10pt,a4paper]{article}
\usepackage[utf8]{inputenc}
\usepackage[spanish]{babel}
\usepackage{amsmath}
\usepackage{amsfonts}
\usepackage{amssymb}
\usepackage{graphicx}
\usepackage[left=2cm,right=2cm,top=2cm,bottom=2cm]{geometry}
\author{Lisandro Alvarez}

\begin{document}
\section{Ejercicio 1}
\subsection{item a}
Buscamos en el reference manual seccion 11.5: Port Control and Interrupt. La direccion es \texttt{4004\_9030}, en Hexa.
\subsection{item b}
En la seccion 11.5.1 del RM esta la estructura del PCR. Es el bit 24.
\subsection{item c}
\subsubsection{PDDR}
Seccion 55.2.6 indica los valores que toma el  \texttt{GPIOx\_PDDR} (Port Data Direction Register) luego del reset. setea a 0: General Purpose Input.

\subsubsection{PCR->SRE}
Sección 11.5.1 indica como quedan los bits del PCR luego del reset. SRE (Slew Rate Enable) varia según el puerto. Nos fijamos la especificación del puerto B en 
0: Fast Slew Rate
1: Slow Slew Rate

Seccion 2.2.2 indica SRE segun puerto luego del reset. El estado es Disable, es decir, el estado de SRE es 1: Fast Slew Rate (no ralentiza la salida).

\subsubsection{PCR->Pull}
el bit PS indica el tipo de pull (pullup o pulldown) y el bit PE indica si esta habilitada o no la opcion de pull interno para ese pin. Luego del reset se pone PE en Disabled y PS pulldown

PE
\begin{itemize}
\item 0: disabled
\item 1: ebabled
\end{itemize}

PS
\begin{itemize}
\item 0: pulldown
\item 1: pullup
\end{itemize}


PS:
0:

\subsubsection{PDOR}

Seccion 55.2.1 indica que valores toma el \texttt{GPIOx\_PDOR} (Port Data Output Register) luego de un reset. Setea todos los pines a 0.

\subsubsection{PDIR}
Seccion 55.2.5 indica que valores toma el \texttt{GPIOx\_PDIR} (Port Data Input Register) luego del reset. Setea todos los pines a 0.

\section{Ejercicio 2}
Se encuentra en el archivo \texttt{MK64F12.h}. 
Tiene campos reservados porque el RM indica que hay un espacio o salto de direcciones de 'memoria' por ejemplo entre los registros GPCHR y ISFR de un puerto (en este caso nos fijamos en la datasheet del puerto C, pero se repite para todos los puertos).

\end{document}